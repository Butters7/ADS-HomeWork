\documentclass[tikz, margin=5mm]{standalone}
\usepackage[utf8]{inputenc}
\usepackage[T2A]{fontenc}
\usepackage{amsmath}
\usepackage{tikz}
\usetikzlibrary{shapes.geometric, arrows}

\tikzstyle{startstop} = [rectangle, rounded corners, minimum width=3cm, minimum height=1cm, text centered, draw=black, fill=red!35]
\tikzstyle{io} = [trapezium, trapezium left angle=70, trapezium right angle=110, minimum width=3cm, minimum height=1cm, text centered, draw=black, fill=blue!20]
\tikzstyle{process} = [rectangle, minimum width=3cm, minimum height=1cm, text centered, draw=black, fill=orange!20]
\tikzstyle{decision} = [diamond, aspect=2, minimum width=3cm, minimum height=1cm, text centered, draw=black, fill=green!20]
\tikzstyle{arrow} = [thick,->]

\begin{document}

\begin{tikzpicture}[node distance=2cm]
  \node (start) [startstop] {Старт};
  \node (input) [io, below of=start] {Задана функция $f(x)$, интервал $[a, b]$ и погрешность $\varepsilon$};
  
  \node (dec1) [decision, below of=input] {$f(a) \cdot f(b) < 0$ ?};
    \node (proc0) [process, left of=dec1, text width=3cm, xshift=-4cm] {Вывести ошибку о невозможности найти корень на заданном интервале};
    % \node (stop1) [startstop, below of=proc0] {Стоп};

  \node (proc1) [process, below of=dec1] {Вычислить $c = \dfrac{a + b}{2}$};
  
  \node (dec2) [decision, below of=proc1] {$\left| b - a \right| > \varepsilon$ ?};
    \node (proc2) [process, right of=dec2, text width=3cm, xshift=+4cm, yshift=-1.5cm] {Вернуть значение $c$ как найденный корень};
    \node (stop2) [startstop, below of=proc2, yshift=+0.5cm] {Стоп};

  \node (dec3) [decision, below of=dec2, yshift=-3cm] {$f(c) = 0$ ?};
  \node (dec4) [decision, below of=dec3, yshift=-0.5cm] {$f(a) \cdot f(c) < 0$ ?};

  \node (proc3) [process, below of=dec4, xshift=-3cm] {Присвоить $b$ значение $c$};
  \node (proc4) [process, below of=dec4, xshift=+3cm] {Присвоить $a$ значение $c$};


  \draw [arrow] (start) -- (input);
  \draw [arrow] (input) -- (dec1);
  \draw [arrow] (dec1)  -- node[xshift=0.4cm, yshift=+0.3cm]{Нет} (proc0);
  \draw [arrow] (proc0) |- (stop2);
  \draw [arrow] (dec1) -- node[xshift=+0.4cm]{Да} (proc1);
  \draw [arrow] (proc1) -- (dec2);
  \draw [arrow] (dec2) node[anchor=center, xshift=+2cm, yshift=+0.3cm]{Нет} -| (proc2);
  \draw [arrow] (proc2) -- (stop2);
  \draw [arrow] (dec2) -- node[anchor=south, xshift=+0.4cm, yshift=1.3cm]{Да} (dec3); 
  \draw [arrow] (dec3.east) node[anchor=center, xshift=+0.3cm, yshift=+0.2cm]{Да} |- (proc2);
  \draw [arrow] (dec3) node[anchor=north, xshift=-0.4cm, yshift=-0.6cm]{Нет} -- (dec4);
  \draw [arrow] (dec4) node[anchor=center, xshift=-1.5cm, yshift=-0.6cm]{Да} -- (proc3);
  \draw [arrow] (dec4) node[anchor=center, xshift=+1.5cm, yshift=-0.6cm]{Нет} -- (proc4);
  \draw [arrow] (proc3) |- (proc1);
  \draw [arrow] (proc4) |- (proc1);

\end{tikzpicture}

\end{document}
