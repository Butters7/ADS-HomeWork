\documentclass[tikz, margin=15mm]{standalone}
\usepackage[utf8]{inputenc}
\usepackage[russian]{babel}
\usetikzlibrary{shapes.geometric, arrows}

\tikzstyle{startstop} = [rectangle, rounded corners=2.5mm,     minimum width=20mm, minimum height=10mm,     text centered, draw=black, fill=red!35]
\tikzstyle{io} = [trapezium, trapezium left angle=70, trapezium right angle=110, minimum width=3cm, minimum height=1cm, text centered, draw=black, fill=blue!30]
\tikzstyle{process} = [rectangle,    minimum width=40mm, minimum height=15mm,    text centered, draw=black, fill=yellow!20]
\tikzstyle{decision} = [diamond, minimum width=3cm, minimum height=1cm, text centered, draw=black, fill=green!30]
\tikzstyle{arrow} = [thick,->,>=stealth]
\begin{document}
\begin{tikzpicture}[node distance=2cm]

% Define the nodes
\node (start) [startstop] {Начало};
\node (init) [process, below of=start] {Инициализация};
\node (loop) [decision, below of=init, yshift=-1.5cm] {Конец?};
\node (fib) [process, below of=loop, yshift=-1.5cm] {Рассчитать $n$-ое число Фибоначчи};
\node (compare) [decision, below of=fib, yshift=-1.5cm] {$n$ равно $x$?};
\node (result) [io, below of=compare, yshift=-1.5cm] {Вывести результат};
\node (update) [process, right of=compare, xshift=5cm] {Обновить переменные};
\node (stop) [startstop, below of=result] {Конец};

% Connect the nodes with arrows
\draw [arrow] (start) -- (init);
\draw [arrow] (init) -- (loop);
\draw [arrow] (loop) -- node[anchor=east] {Да} (fib);
\draw [arrow] (fib) -- (compare);
\draw [arrow] (compare) -- node[anchor=east] {Да} (result);
\draw [arrow] (compare) -- node[anchor=south] {Нет} (update);
\draw [arrow] (update) |- (loop);
\draw [arrow] (result) -- (stop);

\end{tikzpicture}
\end{document}