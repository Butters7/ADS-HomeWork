\documentclass[tikz, margin=3mm]{standalone}
\usepackage[utf8]{inputenc}
\usepackage{tikz}
\usepackage[russian]{babel}

% \hyphenation
\usetikzlibrary{shapes.geometric, arrows, calc}
\tikzstyle{startstop} = [rectangle, rounded corners, minimum width=3cm, minimum height=1cm,text centered, draw=black, fill=red!30]
\tikzstyle{io} = [trapezium, trapezium left angle=70, trapezium right angle=110, minimum width=3cm, minimum height=1cm, text centered, draw=black, fill=blue!30]
\tikzstyle{process} = [rectangle, minimum width=3cm, minimum height=1cm, text centered, draw=black, fill=orange!30]
\tikzstyle{decision} = [diamond, minimum width=3cm, minimum height=1cm, text centered, draw=black, fill=green!30]

\tikzstyle{arrow} = [thick,->,>=stealth]


\begin{document}
    \begin{tikzpicture}[node distance=2cm]
        \node(start)[startstop]{НАЧАЛО};
        \node(loop_start)[process, below of = start, text width = 7cm]{Вычисляем границы поиска: левая граница(left) = 0, правая граница(right) = длина входящего массива(a) - 1};
        \node(max_find)[process, below of = loop_start, text width=7cm, yshift = -1cm]{Находим индекс элемента, который предположительно должен быть нашим искомым числом(key) по формуле: mid = left + (key - a[left]) * (right - left) / (a[right] - a[left])};
        \node(condition_1)[decision, below of = max_find, yshift = -3cm, xshift = -7cm, text width=4cm]{Значение элемента под найденым индексом(a[mid]) < искомого(key)?};
        \node(condition_2)[decision, below of = max_find, yshift = -3cm, text width=4cm]{Значение элемента под найденым индексом(a[mid]) > искомого(key)?};
        \node(condition_3)[process, below of = max_find, yshift = -3cm, xshift = 7cm, text width=4cm]{Возвращаем mid, как индекс искомого элемента};
        \node(proc_1)[process, below of = condition_1, yshift = -2.5cm, text width = 5cm]{Сдвигаем левую границу: left = mid + 1};
        \node(proc_2)[process, below of = condition_2, yshift = -2.5cm, text width = 5cm]{Сдвигаем правую границу: right = mid - 1};
        \node(condition_max)[decision, below of = proc_2, yshift = -3cm, text width=4cm]{Искомое значение находится между левой и правой границей(a[left]<key<a[right])?};
        \node(condition_11)[decision, below of = condition_max, yshift = -5cm, xshift = -7cm, text width=4cm]{Значение элемента, находящегося на левой границе(a[left]) = искомый(key)?};
        \node(condition_21)[decision, below of = condition_max, yshift = -5cm, text width=4cm]{Значение элемента, находящегося на правой границе(a[right]) = искомый(key)?};
        \node(condition_31)[process, below of = condition_max, yshift = -5cm, xshift = 7cm, text width=5cm]{Возвращаем ошибку, говорящую, что элемент не найден};
        \node(proc_11)[process, below of = condition_11, yshift = -2.5cm, text width = 5cm]{Возвращаем left};
        \node(proc_21)[process, below of = condition_21, yshift = -2.5cm, text width = 5cm]{Возвращвем right};
        
        \node(stop)[startstop, below of = proc_21, yshift = -2cm]{КОНЕЦ};

        \coordinate(max_ind) at ($(max_find.north) + (0, 5mm)$);
        % \coordinate(i_cond) at ($(max_find.north) + (0, 5mm)$);

        \draw[arrow](start)--(loop_start);
        \draw[arrow](loop_start)--(max_find);
        \draw[arrow](max_find)--(condition_1.north);
        \draw[arrow](condition_1.east)-- node[anchor = south]{нет}(condition_2);
        \draw[arrow](condition_2.east)-- node[anchor = south]{нет}(condition_3);
        \draw[arrow](proc_2)--(condition_max);
        \draw[arrow](proc_1)--(condition_max.north);
        \draw[arrow](condition_1)--node[anchor = west]{да}(proc_1);
        \draw[arrow](condition_2)--node[anchor = west]{да}(proc_2);
        \draw[arrow](condition_11.east)-- node[anchor = south]{нет}(condition_21);
        \draw[arrow](condition_21.east)-- node[anchor = south]{нет}(condition_31);
        \draw[arrow](condition_11)--node[anchor = west]{да}(proc_11);
        \draw[arrow](condition_21)--node[anchor = west]{да}(proc_21);
        \draw[arrow](condition_max.south)-- node[anchor = west, yshift = -1mm]{нет}(condition_11.north);
        % \draw[arrow](revert_1)--(condition_0);
        \draw[arrow](condition_max.west)--node[anchor = south, xshift = -10mm]{да}+(-8, 0) |-(max_ind);
        \draw[arrow](proc_11)--(stop);
        \draw[arrow](proc_21)--(stop);
        % \draw[arrow](condition_0)-- node[anchor = west]{да}(stop);

    \end{tikzpicture}
\end{document}