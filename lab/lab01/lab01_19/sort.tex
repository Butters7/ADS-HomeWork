\documentclass[tikz, margin=3mm]{standalone}
\usepackage[utf8]{inputenc}
\usepackage{tikz}
\usepackage[russian]{babel}

\usetikzlibrary{shapes.geometric, arrows, calc}
\tikzstyle{startstop} = [rectangle, rounded corners, minimum width=3cm, minimum height=1cm,text centered, draw=black, fill=red!30]
\tikzstyle{io} = [trapezium, trapezium left angle=70, trapezium right angle=110, minimum width=3cm, minimum height=1cm, text centered, draw=black, fill=blue!30]
\tikzstyle{process} = [rectangle, minimum width=3cm, minimum height=1cm, text centered, draw=black, fill=orange!30]
\tikzstyle{decision} = [diamond, minimum width=3cm, minimum height=1cm, text centered, draw=black, fill=green!30]

\tikzstyle{arrow} = [thick,->,>=stealth]


\begin{document}
    \begin{tikzpicture}[node distance=2cm]
        \node(start)[startstop]{НАЧАЛО};
        \node(create)[process, below of = start]{Создаём заданное количество блоков};
        \node(loop_start)[process, below of = create]{i = 0};
        \node(klad)[process, below of = loop_start, text width = 7cm]{Кладём i-й элемент в соответствующую корзину};
        \node(condition_0)[decision, below of = klad, yshift = -1.5cm]{i равен длине массива?};
        \node(plus_i)[process, left of = klad, xshift = -3cm, yshift = -1.5cm]{i = i + 1};


        \node(loop2_start)[process, below of = condition_0, yshift = -2cm]{k = 0};
        \node(sort_bask)[process, below of = loop2_start, text width = 7cm]{Сортируем элементы k-й корзины любым методом сортировки};
        \node(condition_1)[decision, below of = sort_bask, yshift = -2cm]{k равно количеству корзин?};
        \node(plus_k)[process, left of = sort_bask, xshift = -3cm, yshift = -2cm]{k = k + 1};

        \node(loop3_start)[process, below of = condition_1, yshift = -2cm]{j = 0};
        \node(insert)[process, below of = loop3_start, text width = 7cm]{Записываем элементы j-й корзины в массив};
        \node(condition_2)[decision, below of = insert, yshift = -2cm]{j равно количеству корзин?};
        \node(plus_j)[process, left of = insert, xshift = -3cm, yshift = -2cm]{j = j + 1};


        \node(stop)[startstop, below of = condition_2, yshift = -2.5cm]{КОНЕЦ};

        \draw[arrow](start)--(create);
        \draw[arrow](create)--(loop_start);
        \draw[arrow](loop_start)--(klad);
        \draw[arrow](klad)--(condition_0);
        \draw[arrow](loop2_start)--(sort_bask);
        \draw[arrow](sort_bask)--(condition_1);
        \draw[arrow](loop3_start)--(insert);
        \draw[arrow](insert)--(condition_2);

        \draw[arrow](condition_0)-| node[anchor = west,  yshift = 3mm, xshift = 1cm]{нет}(plus_i);
        \draw[arrow](condition_0)--node[anchor = west,  yshift = 1mm]{да}(loop2_start);
        \draw[arrow](plus_i)|- node[anchor = south]{}+(0, 2.5) -| (klad.north);

        \draw[arrow](condition_1)-| node[anchor = west,  yshift = 3mm, xshift = 1cm]{нет}(plus_k);
        \draw[arrow](condition_1)--node[anchor = west,  yshift = 1mm]{да}(loop3_start);
        \draw[arrow](plus_k)|- node[anchor = south]{}+(0, 3) -| (sort_bask.north);

        \draw[arrow](condition_2)-| node[anchor = west,  yshift = 3mm, xshift = 1cm]{нет}(plus_j);
        \draw[arrow](condition_2)--node[anchor = west,  yshift = 1mm]{да}(stop);
        \draw[arrow](plus_j)|- node[anchor = south]{}+(0, 3) -| (insert.north);

    \end{tikzpicture}
\end{document}